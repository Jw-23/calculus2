\section{练习(2025-02-28)}

作业简要给出过程即可
\begin{exercise}\label{exe:1.5}
    求空间3点\(A,B,C\)满足下列关系的充要条件:
\[
\overrightarrow{OA}\times\overrightarrow{OB}+\overrightarrow{OB}\times\overrightarrow{OC}+\overrightarrow{OC}\times\overrightarrow{OA}=0.
\]
\end{exercise}
\begin{exercise}\label{exe:1.1}
    求异面直线 \(L_{1}:\frac{x-1}{0}=\frac{y}{1}=\frac{z}{1}\) 与 \(L_{2}:\frac{x}{2}=\frac{y}{-1}=\frac{z+2}{0}\) 之间的距离.
\end{exercise}

\begin{exercise}\label{exe:1.2}
    已知直线 \(L_{1}:\begin{cases} x=t+1 \\ y=2t-1 \\ z=t \end{cases}\) 与 \(L_{2}:\begin{cases} x=t+2 \\ y=2t-1 \\ z=t+1 \end{cases}\) 求直线\(L_{1}\)与\(L_{2}\)之间的距离.
\end{exercise}

\begin{exercise}\label{exe:1.3}
    求直线 \(L:\begin{cases}2x-4y+z=0,\\ 3x-y-2z-9=0\end{cases}\) 在平面 \(4x-y+z=1\) 上的投影直线方程.
\end{exercise}
\begin{exercise}\label{exe:1.4}
    求过点$P(-3,5,9)$与
\(L_{1}:\begin{cases}y=3x+5\\ z=2x-3\end{cases}\)
和
\(L_{2}:\begin{cases}y=4x-7,\\ z=5x+10\end{cases}\)
都相交的直线方程.
\end{exercise}

\begin{solution}
    \boxed{\textnormal{\ref{exe:1.5} 解答}}

充要条件为:\(A, B, C\) 三点共线。

\textbf{证明:}

\textbf{必要性:} 若 \(A, B, C\) 三点共线,则 \( \overrightarrow{AB} \parallel \overrightarrow{AC} \),故 \( \overrightarrow{AB} \times \overrightarrow{AC} = \vec{0} \)。

又 \( \overrightarrow{AB} = \overrightarrow{OB} - \overrightarrow{OA} \),\( \overrightarrow{AC} = \overrightarrow{OC} - \overrightarrow{OA} \)。

\[
\begin{aligned}
\overrightarrow{AB} \times \overrightarrow{AC} &= (\overrightarrow{OB} - \overrightarrow{OA}) \times (\overrightarrow{OC} - \overrightarrow{OA}) \\
&= \overrightarrow{OB} \times \overrightarrow{OC} - \overrightarrow{OB} \times \overrightarrow{OA} - \overrightarrow{OA} \times \overrightarrow{OC} + \overrightarrow{OA} \times \overrightarrow{OA} \\
&= \overrightarrow{OB} \times \overrightarrow{OC} + \overrightarrow{OA} \times \overrightarrow{OB} + \overrightarrow{OC} \times \overrightarrow{OA} + \vec{0} \\
&= \vec{0}
\end{aligned}
\]
即 \( \vec{OA}\times\vec{OB}+\vec{OB}\times\vec{OC}+\vec{OC}\times\vec{OA}=0 \)。

\textbf{充分性:} 若 \( \vec{OA}\times\vec{OB}+\vec{OB}\times\vec{OC}+\vec{OC}\times\vec{OA}=0 \),则 \( \overrightarrow{AB} \times \overrightarrow{AC} = 0 \),故 \( \overrightarrow{AB} \parallel \overrightarrow{AC} \)。由于 \( \overrightarrow{AB} \) 与 \( \overrightarrow{AC} \) 有公共点 \(A\),所以 \(A, B, C\) 三点共线。

\textbf{答案:} \(A,B,C\)三点共线
\end{solution}

\begin{solution}
\boxed{\textnormal{\ref{exe:1.1} 解答}}

异面直线 \(L_{1}\) 与 \(L_{2}\) 之间的距离公式为:
\[ d = \frac{|\overrightarrow{P_{1}P_{2}} \cdot (\vec{s_{1}} \times \vec{s_{2}})|}{|\vec{s_{1}} \times \vec{s_{2}}|} \]
其中,
\begin{itemize}
    \item \(L_{1}\): 方向向量 \( \vec{s_{1}} = (0, 1, 1) \),已知点 \( P_{1} = (1, 0, 0) \)
    \item \(L_{2}\): 方向向量 \( \vec{s_{2}} = (2, -1, 0) \),已知点 \( P_{2} = (0, 0, -2) \)
    \item \( \overrightarrow{P_{1}P_{2}} = (-1, 0, -2) \)
\end{itemize}

经过计算,异面直线 \(L_{1}\) 与 \(L_{2}\) 之间的距离为:\boxed{1}

\end{solution}

\begin{solution}
    \boxed{\textnormal{\ref{exe:1.2} 解答}}

直线 \(L_1\) 与 \(L_2\) 的方向向量均为 \( \vec{s} = (1, 2, 1) \),故两直线平行。

取 \(L_1\) 上点 \(P_1(1, -1, 0)\),\(L_2\) 上点 \(P_2(2, -1, 1)\),则 \( \overrightarrow{P_1P_2} = (1, 0, 1) \)。

平行线间距离 \( d = \frac{|\overrightarrow{P_1P_2} \times \vec{s}|}{|\vec{s}|} \)。

\( \overrightarrow{P_1P_2} \times \vec{s} = \begin{vmatrix} \mathbf{i} & \mathbf{j} & \mathbf{k} \\ 1 & 0 & 1 \\ 1 & 2 & 1 \end{vmatrix} = (-2)\mathbf{i} + 0\mathbf{j} + 2\mathbf{k} = (-2, 0, 2) \)。

\( |\overrightarrow{P_1P_2} \times \vec{s}| = \sqrt{(-2)^2 + 0^2 + 2^2} = \sqrt{8} = 2\sqrt{2} \)。

\( |\vec{s}| = \sqrt{1^2 + 2^2 + 1^2} = \sqrt{6} \)。

\( d = \frac{2\sqrt{2}}{\sqrt{6}} = \frac{2}{\sqrt{3}} = \frac{2\sqrt{3}}{3} \)。
\end{solution}

\begin{solution}
    \boxed{\textnormal{\ref{exe:1.3} 解答}}

直线 \(L\) 的方向向量为 \( \vec{s} = (9, 7, 10) \),点 \( P = (\frac{18}{5}, \frac{9}{5}, 0) \) 在直线 \(L\) 上。

投影平面法向量 \( \vec{n_{proj}} = \vec{s} \times \vec{n_p} = (17, 31, -37) \),其中 \( \vec{n_p} = (4, -1, 1) \) 为投影平面 \( 4x - y + z = 1 \) 的法向量。

投影平面方程为:\( 17(x - \frac{18}{5}) + 31(y - \frac{9}{5}) - 37(z - 0) = 0 \),化简得 \( 17x + 31y - 37z = 117 \)。

投影直线方程为投影平面与平面 \( 4x - y + z = 1 \) 的交线,即:

\[
\begin{cases}
4x - y + z = 1 \\
17x + 31y - 37z = 117
\end{cases}
\]

\textbf{答案:} 投影直线方程为 \( \begin{cases} 4x - y + z = 1 \\ 17x + 31y - 37z = 117 \end{cases} \)
    
\end{solution}

\begin{solution}
    \boxed{\textnormal{\ref{exe:1.4} 解答}}

    \textbf{解:}

将直线 \(L_{1}\) 和 \(L_{2}\) 转换为参数方程:
\[
L_{1}:\begin{cases}x=t\\ y=3t+5\\ z=2t-3\end{cases} \quad
L_{2}:\begin{cases}x=s\\ y=4s-7\\ z=5s+10\end{cases}
\]
设 \(M(t, 3t+5, 2t-3) \in L_{1}\),\(N(s, 4s-7, 5s+10) \in L_{2}\),则 \( \overrightarrow{PM} = (t+3, 3t, 2t-12) \),\( \overrightarrow{PN} = (s+3, 4s-12, 5s+1) \)。

由 \( \overrightarrow{PM} \parallel \overrightarrow{PN} \) 得 \( \frac{t+3}{s+3} = \frac{3t}{4s-12} = \frac{2t-12}{5s+1} \)。

解得 \( t = -\frac{240}{29} \)。方向向量 \( \vec{v} = \overrightarrow{PM} = (17, 80, 92) \)。

直线方程为:\( \frac{x + 3}{17} = \frac{y - 5}{80} = \frac{z - 9}{92} \)。

\textbf{答案:}  \( \frac{x + 3}{17} = \frac{y - 5}{80} = \frac{z - 9}{92} \)
\end{solution}