\section{向量}

\begin{enumerate}
    \item 在\(\triangle ABC\)中,\(D\)点在边\(AB\)上,且\(AD:AB=\lambda\),用\(\textbf{b},\textbf{c}\)以及\(\lambda\)求\(\overrightarrow{AD}\)的模长.
    \item  对平面向量$\textbf{a}=(a_1,a_2),\textbf{b}=(b_1,b_2)$,证明下列恒等式成立:$$\|\textbf{a}\|^2\|\textbf{b}\|^2=(\textbf{a}\cdot\textbf{b})^2+\|\textbf{a}\times\textbf{b}\|^2$$
    \item  你能否证明对于任意平面向量\(\textbf{a}=(a_1,a_2,\dots,a_n),\textbf{b}=(b_1,b_2,\dots b_n)\),Lagrange 恒等式$$\|\textbf{a}\|^2\|\textbf{b}\|^2=(\textbf{a}\cdot\textbf{b})^2+\|\textbf{a}\times\textbf{b}\|^2$$成立,这个恒等式和Cauchy-Schawtz不等式有什么关系?
    \item \(\textbf{a},\textbf{b},\textbf{c}\)为三个向量, 证明若存在不全为零的常数\(k,l,m\),使得\(k\textbf{a}\times \textbf{b}+l\textbf{b}\times c+m\textbf{c}\times \textbf{a}=0\),则\(\textbf{a}\times \textbf{b},\textbf{b}\times \textbf{c},\textbf{c}\times \textbf{a}\)共线.
    \item 求空间 $3$ 点$A,B,C$满足下列关系的充要条件:$\overrightarrow{OA}\times\overrightarrow{OB}+\overrightarrow{OB}\times\overrightarrow{OC}+\overrightarrow{OC}\times\overrightarrow{OA}=0.$
    
    \item (选做)\(S=\{\textbf{v}_1,\textbf{v}_2\dots,\textbf{v}_n\}\)的元素都是平面向量,且满足\(|\textbf{v}_1|+|\textbf{v}_2|+\cdots+|\textbf{v}_n|=1\),求证存在\(S\)的子集满足\(W\):\(\displaystyle 
    \left|\sum_{\textbf{w}\in W}\textbf w\right|\geq \frac1\pi\).
    \begin{center}\qrcode{https://math.stackexchange.com/questions/1917993/maximum-value-c-s-t-exists-a-subset-s-of-z-1-z-2-ldots-z-n-s-t}
    \end{center}
        
    \item \(\textbf{b}=(x_1,y_1,z_1),\textbf{c}=(x_2,y_2,z_2)\),\(\textbf{b},\textbf{c}\)不共线,求同时垂直于\(\textbf{b},\textbf{c}\)的单位向量.
    \item 设 $\mathbf{a}, \mathbf{b}, \mathbf{c}, \mathbf{d}$ 是欧氏空间中任给的四个起点为 $O$ 的向量,它们的终点分别为 $A, B, C, D$. 证明这些终点 $A, B, C, D$ 共面的充分必要条件是它们中任意取三个得到的混合积满足恒等式:
$$ [\mathbf{a}, \mathbf{b}, \mathbf{c}] - [\mathbf{b}, \mathbf{c}, \mathbf{d}] + [\mathbf{c}, \mathbf{d}, \mathbf{a}] - [\mathbf{d}, \mathbf{a}, \mathbf{b}] = 0. $$
\end{enumerate}
    
\section{空间解析几何}
\begin{enumerate}
    \item 判断下列直线的位置关系:
    \begin{enumerate}
        \item $\frac{x+1}{3}=\frac{y-1}{9}=\frac{z-2}{1}$ 和 $\frac{x}{-1}=\frac{y-2}{2}=\frac{z-1}{3}$
        \item 
        $\begin{cases}
            x+z+1=0 \\
            x+y+1=0
        \end{cases}$
        和
        $\begin{cases}
            x+y+z=0 \\
            y+z+1=0
        \end{cases}$
    \end{enumerate}

    \item 判断下列平面的位置关系:
    \begin{enumerate}
        \item $x+3y-z-2=0$ 和 $2x+6y-2z-2=0$;
        \item $x+y+3z-4=0$ 和 $x+3y+z-4=0$.
    \end{enumerate}

    \item 判断下列直线与平面的位置关系:
    \begin{enumerate}
        \item 直线 $\frac{x-1}{1}=\frac{y-1}{1}=\frac{z-8}{2}$ 平面 $x+y-z+6=0$;
        \item 直线
        $\begin{cases}
            x+3y-2z+1=0 \\
            2x+4y+2z-2=0
        \end{cases}$,
        平面 $x+2y-z-3=0$.
    \end{enumerate}

    \item 求下列直线的方程:
    \begin{enumerate}
        \item 过点(0,1,-1), 与平面 $x-3y+z-2=0$ 平行, 并且和直线
        $\frac{x+13}{2}=\frac{y-5}{3}=\frac{z}{1}$ 和 $\frac{x-10}{5}=\frac{y+7}{4}=\frac{z}{1}$ 都共面;
        \item 平行于向量 $u=(8,7,1)$, 并与直线
        $\begin{cases}
            3x-2y+2z+3=0 \\
            2x+y+z+1=0
        \end{cases}$
        和
        $\begin{cases}
            2x-y-5=0 \\
            3x-2z+7=0
        \end{cases}$
        都相交;
        \item 过点 $(0,1,-1)$, 与直线
        $\begin{cases}
            2x-y-5=0 \\
            3x-2z+7=0
        \end{cases}$
        和
        $\begin{cases}
            x+5y-10=0 \\
            y+z-3=0
        \end{cases}$
        都共面;
        \item 经过点 $M(2,-1,3)$ 平行于向量 $u=(1,0,3)$;
        \item 经过点 $M(3,-2,1)$, 垂直于平面 $3x+2y-3z+5=0$;
        \item 经过点 $M(0,1,-1)$, 并且与直线
        $\begin{cases}
            3x+2y-5=0 \\
            2x-z+3=0
        \end{cases}$
        正交.
    \end{enumerate}

    \item 求下列平面的方程:
    \begin{enumerate}
        \item 过直线 $\frac{x-1}{2}=\frac{y}{1}=\frac{z}{-1}$ 平行于向量 $u=(2,1,-2)$;
        \item 过 $\frac{x+2}{3}=\frac{y-1}{0}=\frac{z}{1}$ 和原点;
        \item 过直线
        $\begin{cases}
            2x+3y+z-1=0 \\
            x+2y-z+2=0
        \end{cases}$,
        平行于向量 $u=(1,1,-1)$;
        \item 平行于平面 $\pi:6x-2y+3z+15=0$, 并且使得点 $(0,-2,-1)$ 到所作平面和平面 $\pi$ 的距离相等;
        \item 平行于 $ x $ 轴, 经过点 $M_{1}(1,-1,2), M_{2}(2,0,-1).$
    \end{enumerate}

    \item 求点到直线的距离:
    \begin{enumerate}
        \item 点 $M(1,0,2)$, 直线
        $\begin{cases}
            3x-2y-1=0 \\
            x-y-2=1 
        \end{cases}$;
        \item 点 $M(3,10,-1)$ 直线 $\frac{x-1}{2}=\frac{y-5}{10}=\frac{z+1}{-3}$ .
    \end{enumerate}
    \item 已知平面 $\pi$ 过点 $(1,1,2)$, 并在 $ x $ 轴, $ y $ 轴, $ z $ 轴上的截距成等差数列, 又知三截距之和为 12. 求平面的方程.

    \item 在直角坐标系中, 平面 $x+y+z=0$ 与二次曲面 $kxy+yz+xz = 0$ 相交于两条直线 $l_1, l_2$. 求正实数 $k$ 的值, 使得这两条直线的夹角是 $\frac{\pi}{2}$.

    \item 在直角坐标系中, 已知平面 $ax + by + cz = 0$ ($abc \neq 0$) 和二次曲面 $xy+yz+zx=0$ 的交线是两条正交的直线, 证明 $\frac{1}{a}+\frac{1}{b}+\frac{1}{c}=0$.

    \item 设 $M_{i}=(x_{i},y_{i},z_{i}), i=1,2,3$ 是不共线的三个点, 证明
    \[
    \begin{vmatrix}
        x & y & z & 1 \\
        x_1 & y_1 & z_1 & 1 \\
        x_2 & y_2 & z_2 & 1 \\
        x_3 & y_3 & z_3 & 1
    \end{vmatrix}
    =0
    \]
    是这三个点所决定的平面的方程.

    \item 求经过 $ z $ 轴, 并且和平面 $2x+y-\sqrt{5}z-1=0$ 的夹角为 $60^\circ$ 的平面的方程.

    \item 求到平面 $3x-2y-6z-4=0$ 和 $2x+2y-z+5=0$ 距离相等的点的轨迹.
\end{enumerate}

